% Awesome Source CV LaTeX Template
%
% This template has been downloaded from:
% https://github.com/darwiin/awesome-neue-latex-cv
%
% Author:
% Christophe Roger
%
% Template license:
% CC BY-SA 4.0 (https://creativecommons.org/licenses/by-sa/4.0/)

%Section: Project

\sectionTitle{Projects}{\faLaptop}

\begin{projects}
    \projectNoLink
        {Localization from IMU and light data time series}{Dec 2021 - Feb 2022}
        {This project exploits the Luxottica® \textit{I-SEEplatform} to understand if the user that is wearing the device is located outside or inside. In the latter case, we also want to identify in which of the known rooms it is. To complete this task we tried different Deep Learning (DL) models searching for the one that leads to the best performance. We finally identified a Long Short-Term Memory (LSTM) as the method that is at the same time, enough accurate and also light in order to allow real-time applications.}
        {\LaTeX, Python, Pycharm, Pandas}
    	
	% \projectNoLink
    % 	{Implementation of a GAN capable of autonomously generating human faces}{May 2021 – Jul 2021}
    % 	{I have developed and implemented a Generative Adversarial Network (GAN) capable of autonomously generating images of human faces from scratch.}
    % 	{Python, TensorFlow, Google Colab, Visual Studio Code, \LaTeX}
	
	% \project
    % 	{AGROBE - WebAPP designed to manage all the needs of an agricultural factory}{Mar 2021 – Jun 2021}
    % 	{\github{caligola25/agrimgmt}}
    % 	{I've designed and implemented a full stack Java Web Application related to the whole management of an agricultural factory (employees, orders, clients, suppliers and production management system).}
    % 	{CSS, Bootstrap, JavaScript, Jquery and AJAX calls, HTML, Java Servlets, postgreSQL, jsp, REST API with servlets}
				
	%\projectNoLink
    	%{Panoramic images stitching}{Mar 2021 – Jun 2021}
    	%{Using C++ and OpenCV I've designed and implemented an algorithm to stitch together a set of related images to %compose a single panoramic image.}
    	%{C++, OpenCV, \LaTeX}
    	
    %\projectNoLink
    	%{Street elements recognition}{Mar 2021 – Jun 2021}
    	%{C++ project made using the OpenCV library to automatically discover the lanes and street signals}
    	%{C++, OpenCV, \LaTeX}
    	
    %\projectNoLink
    	%{Comparative analysis between the MIOTY system and the NB-IoT one}{May 2021}
    	%{This project aims to provide an overview of the basic principles of Low Power Wide Area Network (LPWAN) technologies for the Internet of Things (IoT), and then give a more in-depth presentation of the innovative MIOTY system and, the widespread NB-IoT one. Beyond that, this project aims to compare these two technologies.}
    	%{\LaTeX}
    	
    % \projectNoLink
    % 	{Analysis of Communication of the NBA and Premier League players around the movement and the racial issue on Twitter in 2020}{Dec 2020 – Feb 2021}
    % 	{I've extracted from Twitter (using the relative APIs) all the tweets posted by NBA and Premier League players during 2020. All these tweets were then analyzed and used to build different types of networks (people's networks and semantic networks) using Python together with NetworkX and Gephi. We used these networks to analyze how the BlackLivesMatter hashtag (and related topics) were spread in the world by the NBA and Premier League players.}
    % 	{Twitter APis, Python, NetworkX, Pandas, Gephi, Google Doc}
    	
    % \projectNoLink
    % 	{Management Database of Production in an Agricultural Equipment Factory}{Oct 2020 – Dec 2020}
    % 	{I've designed from scratch a ~15 tables database for managing the employees, the orders and the production phases of an agricultural factory. I also implemented this database using PostgreSQL and designed a simple javax user interface}
    % 	{PostgreSQL}
    	
    % \projectNoLink
    % 	{Implementation of a "meet-in-the-middle attack" in a non-linear Feistel cipher}{Nov 2020}
    % 	{Using Python, implement a “meet-in-the-middle” attack against the concatenation of two instances of the non-linear Feistel cipher, with different keys k′,k′′, respectively.}
    % 	{Python}
    	
    % \projectNoLink
    % 	{Shazam like application}{Oct 2019 – Dec 2019}
    % 	{Working in MATLAB I developed a system that works similarly to Shazam (in a simplified form) and so can recognize the original audio file given a small sample of it (even in noisy conditions). For doing that I've exploited some key properties of the wavelength and the frequency analysis (using FFT).}
    % 	{MATLAB}

\end{projects}